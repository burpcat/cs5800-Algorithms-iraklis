\documentclass[11pt]{article}
\newcommand{\yourname}{Avinash R. Arutla}
\def\comments{0}
%format and packages
%\usepackage{algorithm, algorithmic}
\usepackage{algpseudocode}
\usepackage{amsmath, amssymb, amsthm}
\usepackage{enumerate}
\usepackage{enumitem}
\usepackage{framed}
\usepackage{verbatim}
\usepackage[margin=1.0in]{geometry}
\usepackage{microtype}
\usepackage{kpfonts}
\usepackage{palatino}
\usepackage{tabto}
\DeclareMathAlphabet{\mathtt}{OT1}{cmtt}{m}{n}
\SetMathAlphabet{\mathtt}{bold}{OT1}{cmtt}{bx}{n}
\DeclareMathAlphabet{\mathsf}{OT1}{cmss}{m}{n}
\SetMathAlphabet{\mathsf}{bold}{OT1}{cmss}{bx}{n}
\renewcommand*\ttdefault{cmtt}
\renewcommand*\sfdefault{cmss}
\renewcommand{\baselinestretch}{1.06}
\usepackage[usenames,dvipsnames]{xcolor}
\definecolor{DarkGreen}{rgb}{0.15,0.5,0.15}
\definecolor{DarkRed}{rgb}{0.6,0.2,0.2}
\definecolor{DarkBlue}{rgb}{0.2,0.2,0.6}
\definecolor{DarkPurple}{rgb}{0.4,0.2,0.4}
\usepackage[pdftex]{hyperref}
\hypersetup{
    linktocpage=true,
    colorlinks=true, % false: boxed links; true: colored links
    linkcolor=DarkBlue, % color of internal links
    citecolor=DarkBlue, % color of links to bibliography
    urlcolor=DarkBlue, % color of external links
}
\usepackage[boxruled,vlined,nofillcomment]{algorithm2e}
\SetKwProg{Fn}{Function}{\string:}{}
\SetKwFor{While}{While}{}{}
\SetKwFor{For}{For}{}{}
\SetKwIF{If}{ElseIf}{Else}{If}{:}{ElseIf}{Else}{:}
\SetKw{Return}{Return}
%enclosure macros
\newcommand{\paren}[1]{\ensuremath{\left( {#1} \right)}}
\newcommand{\bracket}[1]{\ensuremath{\left\{ {#1} \right\}}}
\renewcommand{\sb}[1]{\ensuremath{\left[ {#1} \right\]}}
\newcommand{\ab}[1]{\ensuremath{\left\langle {#1} \right\rangle}}
%probability macros
\newcommand{\ex}[2]{{\ifx&#1& \mathbb{E} \else \underset{#1}{\mathbb{E}} \fi \
left[#2\right]}}
\newcommand{\pr}[2]{{\ifx&#1& \mathbb{P} \else \underset{#1}{\mathbb{P}} \fi \
left[#2\right]}}
\newcommand{\var}[2]{{\ifx&#1& \mathrm{Var} \else \underset{#1}{\mathrm{Var}} \fi \
left[#2\right]}}
%useful CS macros
\newcommand{\poly}{\mathrm{poly}}
\newcommand{\polylog}{\mathrm{polylog}}
\newcommand{\zo}{\{0,1\}}
\newcommand{\pmo}{\{\pm1\}}
\newcommand{\getsr}{\gets_{\mbox{\tiny R}}}
\newcommand{\card}[1]{\left| #1 \right|}
\newcommand{\set}[1]{\left\{#1\right\}}
\newcommand{\negl}{\mathrm{negl}}
\newcommand{\eps}{\varepsilon}
\DeclareMathOperator*{\argmin}{arg\,min}
\DeclareMathOperator*{\argmax}{arg\,max}
\newcommand{\eqand}{\qquad \textrm{and} \qquad}
\newcommand{\ind}[1]{\mathbb{I}\{#1\}}
\newcommand{\sslash}{\ensuremath{\mathbin{/\mkern-3mu/}}}
%mathbb
\newcommand{\N}{\mathbb{N}}
\newcommand{\R}{\mathbb{R}}
\newcommand{\Z}{\mathbb{Z}}
%mathcal
\newcommand{\cA}{\mathcal{A}}
\newcommand{\cB}{\mathcal{B}}
\newcommand{\cC}{\mathcal{C}}
\newcommand{\cD}{\mathcal{D}}
\newcommand{\cE}{\mathcal{E}}
\newcommand{\cF}{\mathcal{F}}
\newcommand{\cL}{\mathcal{L}}
\newcommand{\cM}{\mathcal{M}}
\newcommand{\cO}{\mathcal{O}}
\newcommand{\cP}{\mathcal{P}}
\newcommand{\cQ}{\mathcal{Q}}
\newcommand{\cR}{\mathcal{R}}
\newcommand{\cS}{\mathcal{S}}
\newcommand{\cU}{\mathcal{U}}
\newcommand{\cV}{\mathcal{V}}
\newcommand{\cW}{\mathcal{W}}
\newcommand{\cX}{\mathcal{X}}
\newcommand{\cY}{\mathcal{Y}}
\newcommand{\cZ}{\mathcal{Z}}
%theorem macros
\newtheorem{thm}{Theorem}
\newtheorem{lem}[thm]{Lemma}
\newtheorem{fact}[thm]{Fact}
\newtheorem{clm}[thm]{Claim}
\newtheorem{rem}[thm]{Remark}
\newtheorem{coro}[thm]{Corollary}
\newtheorem{prop}[thm]{Proposition}
\newtheorem{conj}[thm]{Conjecture}
\theoremstyle{definition}
\newtheorem{defn}[thm]{Definition}
\newcommand{\instructor}{Iraklis Tsekourakis}
\newcommand{\hwnum}{1}

\newcommand{\mymath}[1]{%
\begin{equation}
#1
\end{equation}
}

%\newcommand{\hwdue}{Wednesday, January 27

\newtheorem{prob}{}
\newtheorem{sol}{Solution}
\definecolor{cit}{rgb}{0.05,0.2,0.45}
\newcommand{\solution}{\medskip\noindent{\color{DarkBlue}\textbf{Solution:}}}
\begin{document}
{\Large
\begin{center}{CS5800: Algorithms} --- \instructor \end{center}}
{\large
\vspace{10pt}
\noindent Homework~\hwnum \vspace{2pt}%\\
%Due :~\hwdue
}
\bigskip
{\large \noindent Name: \yourname }
\vspace{15pt}

 % Solutons

% \begin{prob} \textbf{(x points)}
% \end{prob}

% \solution
% \begin{enumerate}
%     \item $n^2 + 7n + 1$ is $\Omega(n^2)$ \\
%     \solution \\

%     \item \textbf{Some math equation here} \\
%     \solution \\
% \end{enumerate}

 
\begin{prob} \textbf{(18 points)}
\end{prob}

% \solution
\begin{enumerate}
    \item $n^2 + 7n + 1$ is $\Omega(n^2)$ \\
    \solution \\
    To prove that $n^2 + 7n + 1$ is $\Omega(n^2$), \\

    We first need to prove the condition that \\

    f(n) $\geq c.g(n) $ \\

    where f(n) = $n^2 + 7n + 1$, g(n) = $n^2$ (Since $n^2$ is the highest power in f(n) ) \\

    Assuming, \\
    c = 1, we can write

    \[n^2 + 5n + 1 \geq n^2\]
    \[7n+1 \geq 0\]
    \[7n \geq -1 \]
    \[n \geq -1/7\]

    For $n_0$ = 1, \\

    \[n^2 + 7n + 1 \geq n^2\]
    \[1 + 7 + 1 \geq 1^2\]
    \[9 \geq 1\]

    Here, for $n_0$, for all values greater than 0, the f(n) will stay positive. \\
    Therefore at c=1, $n_0$=1, the above equation holds true.

    \item \textbf{$3n^2 + n - 10$ is $O(n^2)$} \\
    \solution \\
    To prove that $3n^2 + n - 10$ is $O(n^2)$ \\

    We need to satisfy that f(n) $\leq$ c.g(gn) for all $n \geq n_0$ \\

    where f(n) = $3n^2 + n - 10$ , g(n) = $n^2$ \\
    
    Here we need to prove that $3n^2 + n - 10$ $\leq$ $c.n^2$ \\

    Since the function $c.g(n)$ must be greater than f(n), we need to cvhoose something which is greater than $3n^2$. \\

    So lets assume, c=4 \\

    \[3n^2+n-10 \leq 4n^2\]
    \[n-10 \leq n^2\]

    Here we can assume that for any value of $n_0 \geq 1$, the positivbe condition holds. \\

    Proof 
    \[n=1 -> 1-10 \leq 1^2 = -9 \leq 1 \]
    \[n=2 -> 2-10 \leq 4 = -8 \leq 4 \]
    \[n=1 -> 3-10 \leq 9 = -7 \leq 9 \]
    \[n=1 -> 4-10 \leq 16 = -6 \leq 16 \]

    Therefore, for all conditions, we could say

    \[ 3n^2+n-10 is O(n^2) \]

    \item \textbf{$n^2$ is $\Omega(nlogn)$} \\
    \solution \\
    To prove that $n^2$ is $\Omega(nlogn)$, we need to prove that $f(n) \geq c.g(n)$ for all $n \geq n_0$ \\

    Here, $f(n) = n^2$, $g(n) = n.logn$ \\

    So, to prove that $n^2 \geq c.(nlogn)$ \\
    Lets assume c=1 \\

    $n^2 \geq 1.(nlogn)$ \\
    $n^2 \geq nlogn$ \\

    From the above equations, we can clearly observe that $n^2$ grows faster than $nlogn$.\\

    Lets assume $n_0=1$
    \[ 1 \geq 1.log1 \]
    \[ 1 \geq 1 \]

    Lets assume $n_0=2$
    \[ 2^2 \geq 2.log2 \]
    \[ 4 \geq 2 \]

    Lets assume $n_0=1$
    \[ 3^2 \geq 3.log3 \]
    \[ 9 \geq 3 \]

    Therefore we can clearly observe that, at $c=1$ and $n_0=2$, the above expression holds true

\end{enumerate}

\begin{prob} \textbf{(20 points)}
\end{prob}
\begin{enumerate}
    \item $T(n)= 2T(n/2)+b$ if $n\geq1$,else 1 if $n=1$ \\
    \solution \\

    The recurrence relation is as follows \\
    \[ T(n) = 2T(\frac{n}{2} + b) \]

    2nd iteration
    \[ T(n) = 2[2T\frac{n}{2^2} + b] + b = 2^2T[\frac{n}{2^2}] + 2b + b \] 

    3rd iteration
    \[ T(n) = 2^2[2T\frac{n}{2^3} + b] + 3b = 2^3T[\frac{n}{2^3}] + 7b \] 

    4th iteration
    \[ T(n) = 2^3[2T\frac{n}{2^4} + b] + 7b = 2^4T[\frac{n}{2^4}] + 15b \] 

    From the above equation, we can understand the pattern that is observed,

    \[ T(n) = 2^kT\frac{n}{2^k} + (2^k-1)b \]

    Since the condition is, \\
    T(1) = 1, we stop when \\
    $ \frac{n}{2^k} = 1$ \\
    $n=2^k$ \\
    $ k = log_2 n$ \\

    Plugging this back into the equation we get,
    \[ T(n) = 2^(log_2 n) + \frac{n}{2^(log_2 n)} + (2^(log_2 ^n - 1)) + b\]
    \[ = 2^(log_2 n)(1) + (n-1)b \]
    
    \[ T(n) = n(1) + (n-1)b \]

    Asympotitc notations are as follows \\

    Big-O : The dominant term here is bn, bt sicne we ignore the constants it is O(n) \\
    Big-$\Omega$ : Sinmce it atleast grows linearly with n, the lower bound notattion is $\Omega(n)$\\

    \item $T(n)= T(n-1)+n+b$ if $n\geq1$,else c if $n=0$ \\
    \solution \\
    Solving for the above equation, T(c) = 0 \\

    \[ T(n-1) = T(n-2) + (n-1) + b \]

    1st Iteration
    \[ T(n-1) + n + b if n>1\]

    2nd Iteration
    \[ T(n-2)+T(n-1) + n + b + b \]

    3rd Iteration
    \[ T(n-3)+T(n-2)+T(n-1) + n + 3b\]

    4th Iteration
    \[ T(n-4)+T(n-3)+T(n-2)+T(n-1) + n + 4b\]

    The pattern here is,
    \[ T(n) = T(n-k) + \sum_{i=1}^{k} (n-k-i) +kb \]

    For the base case, where T(n-k) =0,\\
    n-k=0\\
    n=k at n=0\\

    \[ T(n) = T(n-n) + \sum_{i=1}^{n} (n-n-i) + nb \]
    \[ T(n) = c + \sum_{i=1}^{n} (-i) + nb \]
    \[ T(n) = c + \frac{n(n+1)}{2} + nb \]
    \[ T(n) = c + \frac{n^2}{2} + \frac{n}{2} + nb \]
    \[ T(n) = \frac{n^2}{2} + \frac{n}{2} + nb + c \]

    From here, we can understanmd that order of growth is $O(n^2)$

\end{enumerate}

 
\begin{prob} \textbf{(20 points)}
\end{prob}
\begin{enumerate}
    \item $T(n) = T(n-3) + 3logn$ \\
    \solution \\
    Our guess here is that, T(n) = (nlogn) \\

    Here we need to show that $T(n) \leq cnlog$

    \[ T(n) \leq c(n-3)log(n-3) + 3logn \]
    \[ T(n) \leq cnlog(n-3) -3clog(n-3) + 3logn \] 

    Here, we consider -3 to be negligible compared to log(n) \\

    So, $log(n) > log(n-3)$ \\

    So we can write here that, \\

    \[ T(n) \leq cnlog(n) -3clogn + 3logn \]

    Therefore, $T(n) \leq cnlogn(n$) [Neglecting all the coeffecients] \\

    Here, we can write, 
    \[ -3clogn + 3logn \leq 0 \]
    \[ 3logn(1-c) \leq 0 \]
    \[ c\geq 1\]

    Therefore, we proved that for a constant $c \geq 1$, $T(n) \leq cnlog(n)$, and therefore\\
    $T(n) = O(logn)$

    \item $T(n) = 4T(\frac{n}{3}) + n$ \\
    \solution \\
    Here our guess is $T(n) = O(nlog_3 4)$ \\

    We need to show that, $T(n) \leq cn^{(\log_{3} 4)}$ \\

    Given that, $T(n) = 4T(n/3) + n$ \\

    We can write,
    \[ T(n) \leq cn^{(\log_{3} 4)} \]
    \[ 4.T(n) \leq c\frac{n}{3}^{(\log_{3} 4)} +n\]
    \[ 4.T(n) \leq c\frac{n^{(\log_{3} 4)}}{3^{(\log_{3} 4)}} +n\]
    \[ 4.T(n) \leq c\frac{n^{(\log_{3} 4)}}{4} +n\]
    \[ T(n) \leq cn^{(\log_{3} 4)} +n\]

    Here, we need to prove that $ T(n) \leq c[n^{log_3 4}] + n$, this will hold for an appropriate choice of c, such that the whole recurrence
    relation holds for $T(n) \leq cnlog^(\frac{3}{4})$

    So we will write,
    \[ T(n) \leq cn^{(log_3 4)} \leq cn^{log_3 4} \]

    So, we need to know a value of c for which $n^{log_3 4}$ will overshadow n, so we can write

    \[ cn^{log_3 4} \geq n \]
    \[ c \geq n^{1- log_3 4} \]

    Here we can assume that $1-log_3 ^4 \leq 0$, since $log_3 ^4 > 1$

    So by choosing a value of c, which ios sufficiently greater, the inequality holds

    \[ 1 - log_3 ^4 \leq 0 \]
    \[ log_3 ^4 \geq 1 \]

    Therefore proved.
\end{enumerate}

\begin{prob} \textbf{(20 points)}
\end{prob}
\solution \\

Psuedocode \\

Sort(A,n) : \\
\tabto{2cm} if n $\leq$ 1 : \\
\tabto{3cm} return
\tabto{2cm} Sort(A, n-1)
\tabto{2cm} for 2 to n:
\tabto{3cm} key = A[n]
\tabto{3cm} i = n-1

\tabto{3cm} while i $>$ 0 and A[i] $>$ key:
\tabto{4cm} A[i+1] = A[i]
\tabto{4cm} i-=1
\tabto{3cm} A[i+1] = key
\tabto{2cm} return
\\

To calculate the recurrence relation for the worst case,

\[ T(n) = T(n-1) + c(n-1) + d \]

By using iteration method,\\

1st Iteration : \[ T(n) = T(n-1) + c(n-1) + d \]

2nd Iteration : \[ T(n) = T(n-2) + c(n-2) + d + c(n-1) + d \]

3rd Iteration : \[ T(n) = T(n-3) + c(n-3) + d c(n-2) + d + c(n-1) + d + \]

(n-1)th Iteration : \[ T(n) = T(1) + c(n-(n-1))+ c(n-(n-2)) + ... + (n-1).d \]

\[ T(n) = T(1) + c(1+2+3+4..+(n-1)) + ... + (n-1).d \]

\[ T(n) = T(1) + c(\frac{n(n-1)}{2}) + (n-1).d \]

\[ T(n) = k + c(\frac{n(n-1)}{2}) + (nd-d) \]

\[ T(n) = k + \frac{cn^2}{2} - \frac{cn}{2} + (nd-d) \]

We combine, k, $-\frac{cn}{2}$ , -d to a single term to parse the logic better

\[ T(n) = \frac{cn^2}{2} + nd + a \]

where a = $k - \frac{cn}{2} -d$ \\

Therefore, the expression we can understand that the term $n^2$, which increases faster than the \\
other linear terms. Thus proving that $T(n) = O(n^2)$

\begin{prob} \textbf{(20 points)}
\end{prob}
\solution \\

Here it is given that, 

\[ max(f(n),g(n)) = \theta(f(n)+g(n)) \]
\\
\tabto{0.6cm}Since $\theta$ is the tight bound

We can say that, we need to prove 
\[ max(f(n),g(n)) = \Omega(f(n)+g(n)) \]
\[ max(f(n),g(n)) = O(f(n)+g(n)) \]

We can also write that as, 
\[ 0 \leq c_1(f(n)+g(n)) \leq max(f(n),g(n)) \leq c_2(f(n)+g(n))\]

Here the functions are asympotitically non negative, So for some $n_0 > 0$, $f(n) \geq 0$, $g(n) \geq 0$, there will exist \\

$n \geq n_0$

So we can write, \\
\[ f(n) + g(n) \geq max(f(n),g(n)) \]

Here, we know that,

\[ f(n) \leq max(f(n),g(n)) \]
\[ g(n) \leq max(f(n),g(n)) \]
\\

So, \\

$f(n) + g(n) \leq max(f(n),g(n)) + max(f(n),g(n))$ \\

$f(n) + g(n) \leq 2.max(f(n),g(n))$ \\

$\frac{1}{2}(f(n) + g(n)) \leq max(f(n),g(n))$ \\

So here we can write, 

\[ 0 \leq \frac{1}{2}(f(n)+g(n)) \leq max(f(n),g(n)) \leq (f(n)+g(n)) for n \geq n_0\]

So we can say that \\

\[ (f(n),g(n)) = \theta(f(n) + g(n)) \] 

there will exist $c_1 = \frac{1}{2}$, $c_2 = 1$
\begin{prob} \textbf{(20 points)}
\end{prob}
\solution \\

To prove that $2^{(n+1)}$ = $O{(2^n)}$

Here, we find constants c and $n_0$, such that $\epsilon$ $n \geq n_0$


\[2^{n+1} \leq c.2^n \]
\[2^n . 2 \leq c.2^n \]

Lets assume c=3 and $n_0 =1$, so we can say

\[ 2 \leq 3\]

The above inequality holds true for all $n\geq 1$ and $c \geq 3$. Thus the condition for O is satisfied.

To prove $2^2n$ = $O(2^n)$

We need to prove that, there are positive constants, c and $n_0$ such that $\epsilon$ $n \geq n_0$, $f(n) \leq c.g(n)$

So we can write

\[ 2^2n \leq c.2^n \]

Dividing by $2^2n$

\[ \frac{2^2n}{2^2n} \leq \frac{c.2^n}{2^2n} \]
\[ 1 \leq \frac{c.2^n}{2^2n} \]
\[ 1 \leq \frac{c}{2^n} \]
\[ 2^n \leq c \]

Here, \[ n \leq log_2 c \]

This condition holds true if and only if $n \leq log_2 c$ \\

Since, $2^2n$ grows exponentially, and $2^n$ grows linearly according to n, we cannot say that $2^2n$ will have $O(2^n)$ \\

So, $2^{2n} != O(2^n)$


\end{document}